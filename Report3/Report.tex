\documentclass[a4paper,10pt]{article}
\usepackage[utf8x]{inputenc}
\usepackage{graphicx}
\usepackage{geometry}

\geometry{
body={160mm,230mm},
left=25mm,top=25mm,
headheight=25mm,headsep=7mm,
marginparsep=4mm,
marginparwidth=20mm,
footnotesep=50mm
}

%opening
\title{Software Architectures
Assignment 3: Service Oriented Architectures}
\author{Dehouck Samuel, Delhaye Quentin}

\begin{document}

\maketitle

\section{Introduction}
This assignment was about adding a new feature to the \texttt{web\_portal} application introduced in the previous assignment.
The new functionnality was to query the \texttt{LibrarySearch}web service, consisting in two remote databases.
The \textsc{bpel} process had to be optimized as well by parallelizing what could be.


\section{BPEL Processes}

\subsection{Parallelization}
Before any modification of the process, the different actions done in the main sequence were:
\begin{itemize}
 \item receiveInput: receives input from the requester.
 \item PrepareResponse: prepares the response variable.
 \item AssignSearchRequest: assigns the parameters.
 \item InvokeSearchBooks: invokes the search in the SoftLab library. 
 \item AssignSearchRequest: assigns the parameters.
 \item InvokeSearchForBooks: invokes the search in the National library.
 \item AssignResultSoftLib: assigns the result to the response.
 \item replyOutput: sends the response to the requester.
\end{itemize}

As we can see, the first two actions need to be done sequentially: we need to receive the input first before preparing the response. The same goes for the last two: we need to prepare the response before sending it.
The last two pairs of \textit{Invoke-Assign} are unrelated and their execution can be separated.
In order to parallelize these two couples, we need to use flows. A flow is declared using the tag \texttt{\textless bpel:flow\textgreater}, in which all the declared sequences (using \texttt{\textless bpel:sequence\textgreater})will be executed at the same time.

A graph representing the execution can be found on figure~\ref{fig:graph_seq}

%\vspace{2cm}
 \begin{figure}[h]
	 \centering
 \includegraphics[width=8cm]{diagsequence.png}
 \caption{Execution of the BPEL process.}
 \label{fig:graph_seq}
\end{figure}




\subsection{Data Structure Strategy}
In this table, we describe how books are defined in each library. \\

\begin{center}
\begin{tabular}{|c|c|c|}
\hline
  & National Library & SoftLab Library \\
  \hline
  Author & String & String \\
  \hline
  Date & DateTime & Int \\
  \hline
  ISBN & String & Int \\
  \hline
  Language & String & Language \\
  \hline
  Publisher & String & String \\
  \hline
  Title & String & String \\
  \hline
\end{tabular}
\end{center}
\vspace{0.8cm}

 In order to ease the use of the \textit{LibrarySearch} by additional services, we could normalize the data. There are two ways:
 \begin{itemize}
  \item We could use the types defined by the National Library. But in order to convert the data from the SoftLab Library, we would need to change the year in a full date (day, month and year).
  Some false informations would then be created. The transformation from a \textit{String} to an \textit{Int} would be made without any loss but from a \textit{String} a type \textit{Language}, it would depend on 
  how \textit{Language} is implemented.
  \item We could use the types defined by the SoftLab Library but we would then loose some informations about the date for instance.
 \end{itemize}
 
 As we can see, both choices are valid and the choice would depend on the case.



\section{Integration with Legacy Software}
In order to access the remote web services, we added a new package \texttt{softarch.portal.db.LibrarySearch} containing two classes: \texttt{DatabaseRemote} and \texttt{RegularDatabaseRemote}.
Those are used in the constructor of \texttt{DatabaseFacade} by initializing a regular remote database, no matter what local database configuration has been chosen.

On top of that, the \texttt{findRecords()} method had to be modified.
It now searches in both local and remote databases, and then concatenates the respective results into one single list of \texttt{Book} to return.

Our \texttt{RegularDatabaseCSV} class had to be modified as well.
Indeed, the \texttt{findRecords()} method threw a \texttt{DatabaseException} by default, since that feature was not implemented yet.
It now returns an empty \texttt{ArrayList}, no books being stored in that local database.

The data mapping from \texttt{librarysearch.soft.Book}\footnote{From the client code generated from the \textsc{wsdl} specification.} to \texttt{softarch.portal.data.Book} is done in \texttt{RegularDatabaseRemote.findRecords()} method.
The results from the \textsc{bpel} process are stored in a list of \texttt{Book} of the first type, which is then iterated over and from which each field is extracted and stored in a list of \texttt{Book} of the second type.

The advantage of doing so is that the translation is kept in one place, but adding a new field in the web service book type, or modifying any field would require a modification in this process as well.



\section{Architecture}
The figure~\ref{fig:arch} summarizes the architecture of the subpart of the application handling the databases.
The \texttt{DatabaseFacade} manipulates both local and remote databases, the former by directly fetching the information in the databases, the later by using the \textsc{bpel} process.
The \texttt{DatabaseFacade} is linked to the rest of the application through the \texttt{ApplicationFacade}.
In practice, a user's request will be passed to the \texttt{DatabaseFacade} that will transfer it to the local and remote ressources before returning it to the \texttt{ApplicationFacade}.

\begin{figure}[h]
\centering
\includegraphics[width=\textwidth]{Architecture.png}
\caption{Architecture of the the refactored part of the application.}
\label{fig:arch}
\end{figure}

The figure~\ref{fig:classes} depicts the classes diagram of the refactored part of the application. The new part is on the rightern side, with the two \texttt{..Remote} classes.
As the refactoring of the previous assignment introduced generic interfaces and classes to inherit from, one can see that the rest of the structure of the code did not need to be modified.

\begin{figure}[h]
\centering
\includegraphics[width=\textwidth]{diagrammeClasses.png}
\caption{Classes diagramme of the databases part of the application. The new part from the second assignment is on the right, \texttt{RegularDatabaseRemote} and \texttt{DatabaseRemote}.}
\label{fig:classes}
\end{figure}

\end{document}
